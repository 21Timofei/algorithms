\section{Динамические структуры данных}

\subsection{Стек}

\begin{definicion}
	Стек - структура данных, представляющая из себя упорядоченный набор элементов, в которой добавление новых элементов и удаление существующих производится с одного конца, называемого вершиной стека. Важно, первым из стека удаляется элемент, который был помещен туда последним. В стеке реализуется стратегия «последним вошел — первым вышел» (last-in, first-out — LIFO).
\end{definicion}

	Рассмотрим операции стека :
\begin{enumerate}
	\item \begin{definicion}
			  Push - операция добавления элемента стек.
	      \end{definicion}
	\item \begin{definicion}
			  Pop - операция удаления элемента, находящегося на вершине стека.
		  \end{definicion}
	\item \begin{definicion}
			  Empty - проверка стека на наличие в нем элементов.
		  \end{definicion}
	\item \begin{definicion}
			  Peek - операция, показывающая элемент с вершины стека.
	      \end{definicion}
\end{enumerate}
Реализация стека на n элементов рассходует $O(n)$ памяти, так как нам требуется память только для хранения элементов. Каждая операция будет работаеть за $O(1)$ времени.


\subsubsection{Реализация на массиве}
Заведем массив s на n элементов и переменную count, она нужна для отслежевания индекса элемента с вершины стека.\\

Пример реализации на С++ :

\begin{algorithm}
	\begin{algorithmic}
		\State \textbf{bool} empty() \{

		\State \hspace{1cm} \textbf{return} count == 0;

		\State \}

		\vspace{0.5cm}

		\State \textbf{void} push(\textbf{int} val) \{

		\State \hspace{1cm} count++;

		\State \hspace{1cm} s[count] = val;

		\State \}

		\vspace{0.5cm}

		\State \textbf{void} pop() \{

		\State \hspace{1cm} \textbf{if} (!empty()) \

		\State \hspace{2cm} count- -;

		\State \hspace{1cm} \ \textbf{else} \

		\State \hspace{2cm} \textbf{cout} << "Стек пуст";

		\State \}

	\end{algorithmic}
\end{algorithm}




\subsubsection{Реализация на связном списке}
Для реализации на связном списке необходимо создать структуру, отвечающую за отдельный узел.\\

Пример такой структуры :
\begin{algorithm}
	\begin{algorithmic}
		\State\\
		\textbf{struct} Node \{

		\State \quad \textbf{int} val;

		\State \quad Node *next;

		\State \quad \textbf{Node}();

		\State \quad \textbf{Node}(\textbf{int} val) \{ \textbf{this->val = val;} \}
		\State

		\}
	\end{algorithmic}
\end{algorithm}\\


Каждый узел имеет два поля, первое отвечает за значение, хранящаяся в узле, второе - это указатель на следующий элемент стека.\\\\
Полный пример на C++ :\\

\begin{algorithm}
	\begin{algorithmic}
		\State \textbf{struct} Stack \{

		\State \hspace{1cm} Node *top;

		\State \hspace{1cm} \textbf{Stack}() \{ top = \textbf{nullptr}; \}\\

		\State \hspace{1cm} \textbf{void push}(\textbf{int} val) \{

		\State \hspace{2cm} Node *node = \textbf{new} Node();

		\State \hspace{2cm} node $\rightarrow$ val = val;

		\State \hspace{2cm} node $\rightarrow$ next = top;

		\State \hspace{2cm} top = node;

		\State \hspace{1cm} \}\\

		\State \hspace{1cm} \textbf{void pop}() \{

		\State \hspace{2cm} \textbf{if} (top == \textbf{nullptr})

		\State \hspace{3cm} \textbf{exit}(0);

		\State \hspace{2cm} \textbf{else} \

		\State \hspace{3cm} top = top $\rightarrow$ next;

		\State \hspace{1cm} \}\\

		\State \hspace{1cm} \textbf{int peek}() \{

		\State \hspace{2cm} \textbf{if} (top != \textbf{nullptr})

		\State \hspace{3cm} \textbf{return} top $\rightarrow$ val;

		\State \hspace{2cm} \textbf{else}

		\State \hspace{3cm} \textbf{exit}(1);

		\State \hspace{1cm} \}\\

		\State \hspace{1cm} \textbf{bool empty}() \

		\State \hspace{2cm} \textbf{return} top == \textbf{nullptr};

		\State \};
	\end{algorithmic}
\end{algorithm}




\subsection{Очередь}
\begin{definicion}
	Очередь — это структура данных, добавление и удаление элементов в которой происходит путём операций push и pop соответственно. Притом первым из очереди удаляется элемент, который был помещен туда первым, то есть в очереди реализуется принцип «первым вошел — первым вышел» (first-in, first-out — FIFO). У очереди имеется голова (head) и хвост (tail). Когда элемент ставится в очередь, он занимает место в её хвосте. Из очереди всегда выводится элемент, который находится в ее голове.\\\\ Очередь поддерживает следующие операции:
\end{definicion}

\begin{enumerate}
	\item \begin{definicion}
			  Push - операция добавления элемента в очередь.
	\end{definicion}
	\item \begin{definicion}
			  Pop - операция удаления элемента, находящегося в конце очереди.
	\end{definicion}
	\item \begin{definicion}
			  Empty - проверка очереди на наличие в ней элементов.
	\end{definicion}
	\item \begin{definicion}
			  Size - колличество элементов в очереди.
	\end{definicion}
\end{enumerate}
\subsubsection{Реализация на массиве}
По аналогии c реализацией стека на массиве, для очереди на n элементов выделим массив s размера n, также заведем две переменные tail и head, они будут отслежевать позиции, на которые мы будет добавлять и от куда будем удалять элементы.\\

Пример операций очереди на массиве :
\begin{algorithm}
	\begin{algorithmic}

		\State \textbf{int} size() \{

		\State \hspace{1cm} \textbf{if} (head $>$ tail)

		\State \hspace{2cm} \textbf{return} n - head - tail;

		\State \hspace{1cm} \textbf{else}

		\State \hspace{2cm} \textbf{return} tail - head;

		\State \}

		\vspace{0.5cm}

		\State \textbf{bool} empty() \{

		\State \hspace{1cm} \textbf{return} head == tail;

		\State \}

		\vspace{0.5cm}

		\State \textbf{void} push(\textbf{int} val) \{

		\State \hspace{1cm} \textbf{if} (size() $\neq$ n) \{

		\State \hspace{2cm} s[tail] = val;

		\State \hspace{2cm} tail = (++tail) \% n;

		\State \hspace{1cm} \}

		\State \}

		\vspace{0.5cm}

		\State \textbf{void} pop() \{

		\State \hspace{1cm} \textbf{if} (empty())

		\State \hspace{2cm} \textbf{return};

		\State \hspace{1cm} \textbf{else} \{

		\State \hspace{2cm} \textbf{int} x = s[head];

		\State \hspace{2cm} head = (++head) \% n;

		\State \hspace{1cm} \}

		\State \}

	\end{algorithmic}
\end{algorithm}



\subsubsection{Реализация на связном списке}
Реализация на списке будет отличаться тем, что структура узла будет включать в себя еще одно поле prev, которое будет указателем на предыдущий элемент в очереди.\\


\begin{algorithm}
	\begin{algorithmic}

		\State \textbf{struct} Node \{

		\State \hspace{1cm} \textbf{int} val;

		\State \hspace{1cm} Node *next, *prev;

		\State \hspace{1cm} Node() \{\}

		\State \hspace{1cm} Node(\textbf{int} val) \{

		\State \hspace{2cm} this $\rightarrow$ val = val;

		\State \hspace{2cm} this $\rightarrow$ prev = this $\rightarrow$ next = \textbf{NULL};

		\State \hspace{1cm} \}

		\State \}

		\vspace{0.5cm}

		\State \textbf{struct} Queue \{

		\State \hspace{1cm} Node *head, *tail;

		\State \hspace{1cm} Queue() \{

		\State \hspace{2cm} this $\rightarrow$ head = this $\rightarrow$ tail = \textbf{NULL};

		\State \hspace{1cm} \}

		\vspace{0.5cm}

		\State \hspace{1cm} Node* push(\textbf{int} val) \{

		\State \hspace{2cm} Node *ptr = \textbf{new} Node(val);

		\State \hspace{2cm} ptr $\rightarrow$ next = head;

		\State \hspace{2cm} \textbf{if} (head != \textbf{NULL})

		\State \hspace{3cm} head $\rightarrow$ prev = ptr;

		\State \hspace{2cm} \textbf{if} (tail == \textbf{NULL})

		\State \hspace{3cm} tail = ptr;

		\State \hspace{2cm} head = ptr;

		\State \hspace{2cm} \textbf{return} ptr;

		\State \hspace{1cm} \}

		\vspace{0.5cm}

		\State \hspace{1cm} \textbf{void} pop() \{

		\State \hspace{2cm} \textbf{if} (tail == \textbf{NULL})

		\State \hspace{3cm} \textbf{return};

		\State \hspace{2cm} Node *ptr = tail $\rightarrow$ prev;

		\State \hspace{2cm} \textbf{if} (ptr != \textbf{NULL})

		\State \hspace{3cm} ptr $\rightarrow$ next = \textbf{NULL};

		\State \hspace{2cm} \textbf{else}

		\State \hspace{3cm} head = \textbf{NULL};

		\State \hspace{2cm} \textbf{delete} tail;

		\State \hspace{2cm} tail = ptr;

		\State \hspace{1cm} \}

		\vspace{0.5cm}

		\State \textbf{int} size() \{

		\State \hspace{1cm} \textbf{if} (head $>$ tail)

		\State \hspace{2cm} \textbf{return} n - head - tail;

		\State \hspace{1cm} \textbf{else}

		\State \hspace{2cm} \textbf{return} tail - head;

		\State \}

		\vspace{0.5cm}
		\State \hspace{1cm} \textbf{bool} empty()\{

		\State \hspace{2cm} \textbf{return} size();

		\State \hspace{1cm} \}

		\State \};

	\end{algorithmic}
\end{algorithm}


\subsection{Динамический массив}
\begin{definicion}
	Динамический массив - массив, главным отличием, которого является возможность рассширения по ходу использования.

	Примером динамического массива в C++ является vector. При инициализации vector по-умолчанию начальный размер равен 0. Стратегия расширения проста: при попытке записи в массив нового элемента в момент полного заполнения памяти происходит увеличение размера в 2 раза. При удалении элементов уменьшение размера массива никогда не происходит.

\end{definicion}